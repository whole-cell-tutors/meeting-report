\documentclass[journal,transmag]{IEEEtran}

% ..........................................................................
% Packages, configuration settings, and macro definitions.

\usepackage[pdftex]{graphicx}
\graphicspath{figures}
\DeclareGraphicsExtensions{.pdf,.jpeg,.png}

\usepackage[pdftex,rgb,dvipsnames,svgnames,hyperref,table]{xcolor}

\usepackage[pdftex,breaklinks=true,colorlinks=true,
  bookmarks=false,pdfhighlight=/O,
  urlcolor=DarkBlue,citecolor=DarkRed,linkcolor=DarkBlue]{hyperref}

\usepackage[cmex10]{amsmath}
\interdisplaylinepenalty=2500

\usepackage{amssymb}
\usepackage{amsfonts}
\usepackage{multicol}
\usepackage{enumitem}
\usepackage{accsupp}
\usepackage{array}
\usepackage[caption=false,font=footnotesize]{subfig}
\usepackage{booktabs}
\usepackage{xspace}
\usepackage{soul}
\usepackage{url}
%\usepackage{natbib}
\usepackage{hyphenat}
\usepackage[english]{babel}

% Correct bad hyphenation here
\hyphenation{op-tical net-works semi-conduc-tor}

\usepackage{pdfcomment}
\newcommand{\comment}[3]{\pdfmarkupcomment[markup=Highlight,author={#2}]{#1}{#3}}

% ..........................................................................
% Body.

\begin{document}

\markboth{IEEE Transactions on Biomedical Engineering}%
{2015 whole-cell modeling summer school meeting report}

\title{Combining Standards for \comment{tomorrow's}{Karr}{focus on the future} models: Report of the 2015 whole-cell modeling summer school}

\author{\IEEEauthorblockN{%
Dagmar Waltemath\IEEEauthorrefmark{1},
Falk Schreiber\IEEEauthorrefmark{2}, 
Jonathan R. Karr\IEEEauthorrefmark{3}, and
Other People}%
\IEEEauthorblockA{\IEEEauthorrefmark{1}University of Rostock}
\IEEEauthorblockA{\IEEEauthorrefmark{2}Monash University}%
\IEEEauthorblockA{\IEEEauthorrefmark{3}Department of Genetics \& Genomic Sciences, Icahn School of Medicine at Mount Sinai} %New York, NY 10029 USA\\Email: \href{mailto:karr@mssm.edu}{karr@mssm.edu}}
\thanks{Corresponding author email: dagmar.waltemath@uni-rostock.de}}

\IEEEtitleabstractindextext{%

\begin{abstract}
Computational modeling is an increasingly powerful and important tool for biological discovery, bioengeineering, and medicine. 
Recently, researchers developed the first whole-cell model computational model which represents every individual gene.
However, significant work remains to develop fully complete and accurate models of cells. 
We organized the 2015 whole-cell modeling summer school to teach students the latest cell modeling methods, as well as to bring the computational systems biology community together to assess the need and limitations of our current standards for whole-cell modeling.
We found that a whole-cell modeling standard would accelerate the development of whole-cell models, and that more work is needed to expand our current standards to support whole-cell models.
% One major criticism of these standards is that they lag behind current developments and thereby may not be suitable to fully encode today's models. To overcome this problem, the applicants, under the guidance of the COMBINE effort, will organize a summer school for standard developers and modelers, i.e., users of these standards. The main goal is to explore the expressivity of current standard formats using the example of the famous whole-cell model, which has not yet been encoded in a standard format.
\end{abstract}

% Note that keywords are not normally used for peer review papers.
\begin{IEEEkeywords}
Whole-cell modeling, Systems biology, Simulation, Computational modeling, Standards, Education
\end{IEEEkeywords}
%
}

\maketitle
\IEEEdisplaynontitleabstractindextext
\IEEEpeerreviewmaketitle


\section{Introduction}
% The very first letter is a 2 line initial drop letter followed
% by the rest of the first word in caps.
% 
% form to use if the first word consists of a single letter:
% \IEEEPARstart{A}{demo} file is ....
% 
% form to use if you need the single drop letter followed by
% normal text (unknown if ever used by IEEE):
% \IEEEPARstart{A}{}demo file is ....
% 
% Some journals put the first two words in caps:
% \IEEEPARstart{T}{his demo} file is ....
% 
% Here we have the typical use of a "T" for an initial drop letter
% and "HIS" in caps to complete the first word.
% is a report of the VW Whole-cell Summer School held in Rostock during April, 2015.
\IEEEPARstart{C}{omputational} \comment{modeling is an promising tool for biological discovery, bioengeineering, and medicine.}{Karr}{provide scientific, engineering, and medical motivation for the course, expanding SBML, and for an SBML-encoded whole-cell model} Computational modeling has already been used to identify new metabolic genes \cite{Reed2006}, add metabolic pathways to bacteria \cite{Lee2009}, and identify potential new antimicrobial drug targets \cite{Lee2012}. Computational models also have the potential to enable bioengineers to design entirely new strains of bacteria optimized for industrial tasks such as chemical synthesis, biofuel production, and waste decontamination, as well as to enable clinicians to design individualized medical therapies tailored to each patient's unique genome. Realizing this potential requires more comprehensive and accurate computational models which are capable of predicting cellular behavior from genotype \cite{Macklin2014, Karr2015}.

Recently, researchers at Stanford University developed the first whole-cell model of the gram-positive bacterium \textit{Mycoplasma genitalium} \cite{Karr2012}. The model represents the life cycle of a single cell including the copy number of each metabolite, RNA, and protein species and accounts for every known gene function. The model is comprised of multiple sub-models, each of which was implemented using different mathematical representations such as ordinary differential equations, flux balance analysis, and Boolean rules and trained using different experimental data. 

The \textit{M. genitalium} whole-cell model was implemented in MATLAB, is available open-source under the MIT license, and was extensively documented. This has enabled other researchers to expand the model and use the model for their own research, as well as use the model as a teaching tool in university systems biology courses across the world. 

Although MATLAB is commonly used by academic researchers, MATLAB is proprietary and expensive. In addition, because many of the biological details of the model are intertwined with the MATLAB code, significant domain expertise is required to understand, modify, and expand the model. A more transparent, standardized implementation is needed to enable more researchers to use the existing whole-cell model, as well as to contribute to whole-cell modeling. In turn, this would enable researchers to develop faster and more efficient whole-cell model simulators, more deeply explore whole-cell model predictions, and more rigorously evaluate whole-cell models. \comment{Ultimately, this would accelerate the whole-cell modeling field and enable more researchers to realize the full potential of whole-cell modeling for science, engineering, and medicine.}{Karr}{Its important to emphasize the goal is not just to recode an existing model, but to accelerate the whole-cell modeling field}

The Systems Biology Markup Language (SBML) \cite{hucka2003} and the Cell Markup Language (CellML) \cite{hedley_2001b} are the most commonly used systems biology modeling standards. Both languages can be used to develop a wide variety of models including ordinary differential equation, logical, and flux balance analysis models, and both languages have been used to build hundreds of models. However, neither language supports many of the features needed for whole-cell modeling including large, sparsely occupied state spaces and multi-algorithm simulations. Consequently, further work is needed to expand the SBML and CellML languages to support whole-cell models.

We organized the 2015 whole-cell modeling summer school to train students in whole-cell modeling to expand the whole-cell modeling field, as well as to begin to expand the SBML language to support whole-cell modeling. The goals of the school were three-fold. First and foremost, the goal of the course was to train young researchers how to build whole-cell models including how to develop sub-models of individual cellular pathways using different mathematical formalisms and encode them using SBML, as well as how to combine sub-models into a single multi-algorithm model. The second goal of the course was to identify the features that must be added to the SBML language to support whole-cell models. The third goal of the course was to initiate the recoding of the \textit{M. genitalium} whole-cell model into SBML.

\comment{Here, we summarize the educational and scientific outcomes of the summer school. First, we describe the content and organization of the school. Second, we describe the educational successes of the courses in training new computational systems biologists. Third, we describe the limitations of SBML for whole-cell modeling which we identified through the course. Fourth, we describe our early progress toward developing a SBML-encoded whole-cell model. Lastly, we outline the further research that is needed to expand the SBML language to support whole-cell modeling, as well as the further work that is needed to recode the \textit{M. genitalium} model using SBML.}{Karr}{Outline the rest of the paper here}

\section{The 2015 whole-cell modeling summer school}

\subsection{Format}
\comment{Two scientific invited talks were held at the summer school.}{Karr}{Prior to this provide overview of meeting: (1) talks, (2) division into groups led by tutors, (3) working sessions, (4) summary presentations, (5) final presentations, (6) poster session. Also introduce pre-prep through google hangouts and code sharing through GitHub.} 
The first speaker was Dr. Michael Hucka from the California Institute of Technology, USA. 
Dr. Hucka provided an overview of the COMBINE initiative, COMBINE standards and tools supporting these standards. 
His talked focused on the formats most relevant to whole-cell modeling. 
%Dr. Hucka is one of the founders of SBML and COMBINE.

The second speaker was Dr. Jonathan Karr from the Icahn School of Medicine at Mount Sinai School, USA. 
Dr. Karr provided overviews of the whole-cell modeling field, of his own research toward developing and applying whole-cell models for scientific discovery and engineering, and of the \textit{M. genitalium} whole-cell model which he and his colleagues recently developed.
Dr. Karr concluded his talk by outlining several research projects which his own group is pursuing to expand the scope of whole-cell models and use whole-cell models to engineer faster growing bacteria.
%Dr. Karr's inspiring talk set the basis for re-coding his model.

\comment{The principal organisation of the course was as follows:}{Karr}{These next couple paragraphs are disorganized. First introduce the goal. Then how students were divided into groups to achieve this. Talk about how much time groups had to work. Summarize how the groups were intended to coordinate with each other and provide summaries each day at the plenary session.} 
We worked in eight teams of four to six students and one tutor. 
Each team focused on one part of the whole-cell model. 
The course included three additional ``floating'' tutors which were not assigned to specific teams. These tutors shared their expertise in systems modeling, model documentation, and SBGN with all of the teams.

The goal of each team was to provide a running module of their part of the model, together with the necessary inputs and outputs for the other groups. 
Lastly, one group, called ``Integration'', coordinated the overall integration of all modules.

We chose this format deliberately to have mixed groups of standard developers (mostly the tutors) with modelers (mostly the students). 
At the same time we arranged students in groups so that their expertise matched the specific module, and so that the groups themselves consisted of heterogeneous scientists in terms of education. Lastly, another aspect for building the groups was that we did not want the participants to know each other before (to enhance the network experience for everybody) and internationality of the groups, whenever possible. 
The resulting groups thus were divergent in many aspects. 
However, the surrounding and the frame of the summer school led to a communicative environment. 
All students were open to learn new tools and methods, and they were willing to contribute with their own expertise to the overall task.
Throughout the meeting, each participant presented at least once during a plenum session.

\comment{The schedule was completely free, and each group was left to organise themselves throughout the days.}{Karr}{Moved this information from further observations section}
In the evenings, we had one plenum session to first discuss the day and then summarise the results of that day. 
The evening activities provided room to network, socialise and discuss about the work in a more informal setting, and specifically with participants from other groups. 

At the final meeting, all tutors agreed that the goals of the summer school could not have been achieved without the interdisciplinary mix of expertises. 
The interdisciplinarity helped to see the task from different angles, and certainly it allowed us to try very different approaches to solving a problem (from brute force, to working with design thinking methods). 
In the opening to the summer school, we spoke of the effect of swarm intelligence, and this is what happened during the week of work.

The poster session enabled participants to present their own research. 
27 students presented posters. Many of the students presented projects on whole-cell modeling or other computational systems biology projects.

\subsection{Educating young systems biologists}
One of the major intentions of this summer school was to educate young scientists. 
The whole-cell model is now one of the standard models in computational biology. 
\comment{It is therefore also important for young researchers to be informed about the model, its capabilities, the insights it gives and how it can be used and reused.}{Karr}{Outline more specific skills/knowledge that the course was intended to teach: (1) cell biology, (2) modeling methods: ODE, FBA, Boolean, etc. (3) model integration, (4) standards: SBML, SED-ML, SGBN}

\comment{We were glad to have Dr. Karr present his model at the summer school.}{Outline more specifically what students did learn, and how this was taught through the small group interactions, intro lectures, and popup seminars}
He also taught students about the single sub-models throughout the school. 
As a floating tutor, Dr. Karr was a member of the floating team. He visited the different groups and answered questions both on the modeling and on the biology. 

\comment{The students were well-prepared for the course.}{Karr}{elaborate on what computational systems biology knowledge students already had: (1) modeling, (2) programming, (3) systems biology, (4) standards}
Many had already attended the pre-course classes via Google Hangouts. 
Prior to the course, most of the students had already run the whole-cell model. 

After the summer school, many students reported that they learned a lot about using open-source modeling software. 
In particular, \comment{many}{Karr}{Are there other survey statistic that can be mentioned?} students reported increased understanding of SBML, and better awareness of reproducibility issues.

The course was also a great networking event for both the students and tutors. 
Students had opportunities to connect with other young computational systems biology researchers from around the world.
The poster session provided students an opportunity to share their own work and get valuable feedback from each other, the tutors, and other scientists from the University of Rostock.
Several of the tutors advertised open positions in the labs, as well as upcoming courses and meetings including the April 2016 whole-cell modeling summer school which  Jonathan Karr, Luis Serrano, Maria Lluch-Senar, and Javier Carrera are organizing in Barcelona, Spain.

\subsection{\comment{Setting the goal}{Karr}{I would probably eliminate this section and merge the thoughts about code committed to GitHub with the following section. If the goal was to create a working model, then the goal wasn't met and likely will never be met. Probably very few of the SBML sub-models actually work. None of them have been tested to any degree. And a lot of work remains to integrate them. I definitely wouldn't claim the sub-models will be published by the fall because this is not likely to happen.}}
The goal to encode the whole-cell model was ambitious from the beginning, but the expectations have been more than met. 
All participants dedicated their full time to working on the project, preparing initial results weeks beforehand, and even worked long hours. 
The achieved results are of high quality and the spirit at the summer school was very positive over the whole week. 
With 3 more days on the same project, we would have been able to test and then publish the 28 modules. 
The final state after one week is that all material is there, but the modules need to be tested and integrated. 
Several weeks after the summer school, the activity on the Git project is still high, and it can be expected that we will finish these two tasks before summer. 
The modules shall be published in BioModels Database by autumn 2015. 

\subsection{Progress toward an SBML whole-cell model}
In addition to training young systems biology researchers, the course produced preliminary SBML encoded versions of each of the whole-cell models sub-models.

\comment{We taught standard formats using open standards (COMBINE standards) and open software (COMBINE-compliant software).}{Karr}{This paragraph belongs in the introduction and in the teaching section}
COMBINE, the COmputational Modeling in BIology NEtwork \cite{LeNovere2011} is the umbrella organisation for various standardisation initiatives, including SBML \cite{hucka2003}, CellML \cite{hedley_2001b}, SED-ML \cite{sedml2011}, and SBGN \cite{le2009systems}. 
All formats are \textit{de facto} standards. 
SBML represents networks in biology; CellML represents networks in physiology; SED-ML encodes simulation descriptions; and SBGN encodes the graphical representation of networks. 
Together, these standards allow for the encoding of virtual experiments in biology. 
Many so-called COMBINE-compliant software exist to run these experiments. 
Popular software, used during the summer school, are COPASI \cite{Mendes2009}, BioUML \cite{Kolpakov2006}, VANTED \cite{Rohn2012}, and iBioSim \cite{Stevens2013}.

We enhanced the exchange of information and developed 28 new modules in COMBINE formats that represent the majority of the whole-cell model, and which can be run in open-source software.

An overview of progress in the single modules is shown in Table~1. 
The sub-models are available open-source at \url{http://github.com/dagwa} and \url{https://github.com/whole-cell-tutors/wholecell}. 
The scientific community will benefit from these open-access, reusable versions of the sub-models that make up the model.

% TODO: Insert table of progress here

Further work is needed to integrate these sub-models into a single model. 
We hope to achieve this over the next several months.

\subsection{Limitations of SBML for whole-cell modeling}
\comment{Specific issues}{Karr}{Provide introduction to the limitations of our present standards for whole-cell modeling} included the representation of the \comment{different states of the binding sites}{Karr}{generalize this discussion to all large, combinatorial state spaces} (summing up to 1 Mio in the SBML file), the lack of
representation of arrays in core SBML, the generation of random numbers and the sharing of variables across modules, and how to \comment{determine the order of execution}{Karr}{This is a difference between programming and numerical integration, but not necessarily a limit of sBML} of single processes during simulation.

\comment{The second reason for the delay in encoding is the lack of concepts to represent large arrays of data in SBML.}{Karr}{This isn't necessarily a problem, except that it might make computations slow. The real problem is the inability to represent state large state spaces sparsely. Instead, SBML forces you to enumerate the entire space, which is very inefficient for combinatorially large spaces. Arrays alone would not solve this problem.}
We faced serious problems in representing, for example, variations in DNA binding sites. 
While MATLAB does have concepts for the representation of arrays and vectors, SBML lacks comprehensive ways of encoding these. 
Here, the discussions between modelers, standard developers and tool developers were particularly interesting, as arguments were provided from very different view points.

The positive effect on the standardisation community became immediately visible:
\comment{The summer school offered a great opportunity for standard developers to receive feedback on the usability of COMBINE standards and associated software.}{Karr}{This is a bit confusing. Was this a course or a meeting? Who exactly was getting feedback -- students, tutors?}
This connection, between tool developers, standard developers and modelers is essential and yet usually missing. 
The single groups do have their own meetings and it happens that the communication between them is rather sparse. 
This summer school, however, offered a new channel of communication, through a very concrete task. 
The fact that modelers openly and directly pointed at lacks in existing standards, and that this criticism was converted into positive action on the developers site is for
us the biggest achievement of the summer school. 
\comment{We hope that the summer school will inspire more events like this, where modelers and standard developers work together to solve a biological problem.}{Karr}{These comments at the end of this paragraph could be moved to the conclusion/discussion}

\section{\comment{Lessons learned}{Karr}{Remove the details about funding and focus on lessons that would be useful to other summer school organizers. In particular, focus on the unique format and the advantages of this.}}
From the feedback that we received, we conclude that the summer school was well perceived. 
Most participants thought that the \comment{format}{Karr}{elaborate on this format and how it might be good for other scientific communities} was unusual, but gave them the opportunity to learn a lot. 
However, two participants mentioned that they had not been satisfied with the format, they said that they would have wished for a tighter schedule with more lectures. 
To meet this need, we introduced two break-out sessions with discussions on SBML-specific topics.

\section{Future directions}
\comment{The goal is to construct SBML-encoded versions of all 28 sub-models and to publish them in the BioModels database.}{Karr}{Start with the goals. This discussion should also include expanding SBML and the simulators. Then get into specifics of how this will be achieved.}
This will guarantee a long-time availability of the sub-models. 

The participants of the summer school plan to continue working in their single groups even after the close of the school. 
Specifically, the groups plan to meet virtually via Google Hangouts, to finalise the representation of the modules, the annotations, and the graphical
maps. 

We finished the last day with a long session on ``How to move on'', asking each group to make a plan on how to continue after the course. 
Here, it proved valuable to have all code available in a Git repository that is accessible to everyone. 
All participants of the course said that they would like to finalise the project. 
We hope to publish a scientific paper on the COMBINE-compliant model in autumn with all contributors as co-authors.

In addition, lessons learned from the summer school and weaknesses of COMBINE standards will be discussed during the next standardisation meeting (HARMONY) in April this year. 
It will therefore foster further standard development in systems biology. 

\section{\comment{Conclusion}{Karr}{Broaden conclusion. Circle back to scientific/engineering/medicineg goals}}
\comment{Lastly, we hope that this summer school will become a prime example for modern educational events, showcasing how standard development can happen in close interaction with the end-users.}{Karr}{These sentences belong in the conclusion}
The experiences of this summer school will be reported at this year's HARMONY and COMBINE meetings (as invited talks, opening the two major standardisation meetings for
systems biology). 
We do also aim to give feedback through our European networks, such as EraNet SysBio \cite{ERASysBio2015}, CaSYM \cite{CaSYM2015} and ISBE \cite{Wolkenhauer2009}.

\comment{Jonathan Karr furthermore announced yet another summer school on whole-cell models, taking place in Barcelona in 2016, but with a focus on the theory behind modeling whole cells.}{This doesn't directly relate to the course. I would move this to the conclusion}
We believe that this summer school set the path for a new series of meetings related to whole-cell modeling.

\section*{Acknowledgment}
The course was supported by a grant from the Volkswagen Foundation to DW and FS. 

% Can use something like this to put references on a page
% by themselves when using endfloat and the captionsoff option.
\ifCLASSOPTIONcaptionsoff
  \newpage
\fi

\bibliographystyle{IEEEtran}
\bibliography{IEEEabrv,report}


% biography section
% 
% If you have an EPS/PDF photo (graphicx package needed) extra braces are
% needed around the contents of the optional argument to biography to prevent
% the LaTeX parser from getting confused when it sees the complicated
% \includegraphics command within an optional argument. (You could create
% your own custom macro containing the \includegraphics command to make things
% simpler here.)

\begin{IEEEbiography}[{\includegraphics[width=1in,height=1.25in,clip,keepaspectratio]{photos/waltemath.jpg}}]{Dagmar Waltemath}
\end{IEEEbiography}

\begin{IEEEbiography}[{\includegraphics[width=1in,height=1.25in,clip,keepaspectratio]{photos/schreiber.jpg}}]{Falk Schreiber}
\end{IEEEbiography}

\begin{IEEEbiography}[{\includegraphics[width=1in,height=1.25in,clip,keepaspectratio]{photos/karr.jpg}}]{Jonathan R. Karr}
received the Ph.D. degree in Biophysics and the M.S. degree in Medicine from Stanford University, Stanford, CA, USA in 2014 and the S.B. degrees in Physics and Brain \& Cognitive Sciences from the Massachusetts Institute of Technology, Cambridge, MA, USA in 2006.

He is currently a Fellow at the Icahn School of Medicine at Mount Sinai in New York, NY, USA. His research focuses on the development of comprehensive whole-cell computational models and their applications to bioengineering and medicine.
\end{IEEEbiography}
% or if you just want to reserve a space for a photo:

% \begin{IEEEbiography}{Michael Shell}
% Biography text here.
% \end{IEEEbiography}

% if you will not have a photo at all:
% \begin{IEEEbiographynophoto}{John Doe}
% Biography text here.
% \end{IEEEbiographynophoto}

% insert where needed to balance the two columns on the last page with
% biographies
%\newpage

% \begin{IEEEbiographynophoto}{Jane Doe}
% Biography text here.
% \end{IEEEbiographynophoto}

% You can push biographies down or up by placing
% a \vfill before or after them. The appropriate
% use of \vfill depends on what kind of text is
% on the last page and whether or not the columns
% are being equalized.

%\vfill

% Can be used to pull up biographies so that the bottom of the last one
% is flush with the other column.
%\enlargethispage{-5in}



% ..........................................................................
% End.

\end{document}
